\chapter{Introduction}

	\noindent
	Have you read the abstract? Make sure you have before continuing. People keep telling me they don't understand where I'm going with this. This document was created to respond to the reactions to the TPI... and to give you something that starts to provide data on the hypothesis of the existence of hybrid paradoxes. THIS IS NOT THE COUNTEREXAMPLE. It is just a simple example to encourage you to read a much more complex and lengthy one.\\\\
	
	\noindent
	I can't help it, I write a bit disorderly. I don't know how to solve the problem of you thinking that what I show in each document is ALL I have, and if I try to warn you about this, you say I'm disorganized. Not even this series of documents is EVERYTHING I have. The TPI document has 60 pages. I tried to save time with an internet person who couldn't read Spanish, presenting only what is an 'imperfect injection', without examples, and the definition of the TPI. The result was that person did EVERYTHING I beg you not to do in the introduction. I will attach the document that person sent me 'anonymously' in the appendix chapters. You can't see our conversations, where that person clearly talks about 'rotating sets' and indicates that I would be able to see it if I had a formal education that took all these foolishness out of my head. But 'rotating sets' are my own argument in the DI. The one that was denied to me, and so I built the TPI.\\\\
	
	\noindent
	Remember that I am NOT a mathematician. Still, I have a letter from a professor\footnote{Text in the appendix chapter} saying that him have been unable to find a flaw. If the format or rigor were a REAL flaw, beyond the initial tantrum it may provoke, he would have mentioned it. Well, in fact... he did mentioned it... and still don't consider it a serious flaw. PRECISELY, I am tired of asking for help to be able to write all my material in a more conventional way, and to be able to check it more deeply and expand it. In the absence of resources and qualified people, the only thing left to us is the scientific, philosophical, and moral imperative that truth is more valuable than a lie with a comfortable presentation. At least, in mathematics, I hope. I have gone above and beyond to give my best.\\\\
	
	\noindent
	I keep warning that in front of a blackboard it is more enjoyable, because I can skip unnecessary assumptions and adapt, studying and evaluating the direct feedback from each person.\\\\
	
	\noindent
	Every idea, every concept, has been defended in forums, and after several feedbacks and corrections, they have been checked and considered correct. Every result as well. Except for the point that forces me to add the DI, the interpretation of infinite intersections, which I considered unnecessary after the concept of 'rotating sets' was shot down. We will see it in future documents, or you may have already started to forge it in your mind, in order to deny the TPI, if you have read the COMPLETE document. I keep saying it but it seems it is not being heard (read carefully): if you DECIDE that the subsets of the partition of the ORIGIN SET in the TPI never even come close to the cardinality of the Domain set, that indicates that you firmly believe that there exists a set different from the empty set, subtracting from Domain X Domain the largest subset (NWSP) possible, or ALL of them (remember they are nested). After having tried it with ALL the disjoint subsets of the Origin set in each relation. Although you will be unable to mention a single element within it. This will lead you to IGNORE the UNDENIABLE final result of emptiness, and to rely on the fact that for each subset we always leave an infinite set of elements from NWSP uncovered. A triad of inseparable decisions, as I have been able to confirm in the reactions of several people. If that "leftover" set exists in some strange way, we can choose one or more elements from it... and that is the basis of DI (which we will see in future documents). In DI, the set with supposed cardinality $\aleph_{1}$ is the one that has trouble `eliminating' all of the elements. In DI, we decide to ignore the final empty intersection result, which occurs in the same terms but with the roles reversed, and we only focus on the inability of the set with cardinality $\aleph_{1}$ to get rid of a simple set with cardinality $\aleph_{0}$.\\\\
	
	\noindent
	If you are not going to bother contacting me to ask me questions, or if you are going to read it quickly, skipping parts\footnote{READ THE ABSTRACT!!}, etc... please don't even try. YOU ALREADY KNOW that I have flaws as a writer, getting angry about it instead of sending me an email asking me... I don't know how to describe it. Reading it quickly will lead you to come up with real nonsense (too many previous experiences)... One person underlined almost an entire paragraph telling me that a specific property didn't necessarily work on infinite sets. The sentence that was left ununderlined, right at the beginning of the same paragraph, said something like: "...this only applies to sets of finite cardinality". ASK, read with some care and patience, and if you get angry, remember that I have double-checks, experienced people who can't find the mistake and that I have offered to do it in person thousands of times with a blackboard, and they call me arrogant for it. That doubt or perception of error could be resolved in SECONDS, and there are more than two mathematicians who do NOT consider it an error. Especially with the presence of more mathematicians, we could leave arguments from authority aside, since you could see their faces, not mine. I DON'T MIND ANSWERING A MILLION TIMES, I'm the first non-mathematician here. What bothers me is when communication is cut off due to real nonsense. Sometimes with arguments unworthy of a mathematician. Things like saying that a counterexample does not nullify a theorem, since one must explain separately WHY each of the existing proofs fails. Anyway...\\\\
	
	\noindent
	This week I was able to check something that was causing me brutal frustration. I didn't think it was such a complex argument to understand, but 'someone' has already confirmed to me that it is correct. Even that person did not understand why I insisted on the question, and repeated several times that it was correct. As always, working with anonymous people in forums, with whom I cannot continue working when communication is cut off, and months go by until I find another person.\\\\


	\noindent
	The idea is simple, both constructions, if well constructed, would be equivalent to an impossible injective relationship between $P(\mathbb{N})$ and $\mathbb{N}$. Both the TPI (Unlimited Pair Transfer) and the DI (Inverse Diagonalization) need property A, but in complementary ways. A triad of interpretations about the type of infinite intersections presented in the TPI document. Affirming A would consist of:\\\\
	(A.1): We can ignore the final result of empty. I say ignore because it cannot be denied. In both cases, the infinite intersection has an UNDENIABLE result of empty.\\\\
	(A.2): There is a set that can be empty or not: the 'rotating set'. We are unable to mention a single element within the set because we know with absolute certainty that EVERY ONE of its elements is eliminated from it, from a concrete and calculable step of the infinite intersection. The fact that we are unable to mention a single element within it, WE ARE ALSO GOING TO IGNORE IT... arguing observation A.3.\\\\
	(A.3): The 'rotating set', NWSP in the case of the TPI, the Packs\footnote{It seems silly to remember, but given previous experiences: we will see what the Packs are in future documents} of unique naturals in the DI, always has an infinite cardinal in each step, of the infinite terms of the infinite intersection. This is also undeniable. But we decide NOT TO IGNORE THIS, to say that no subset (subsets of the Origin set in the TPI, families of pairs from the Domain in the DI) ever obliges us to empty the objective set (NWSP or The Packs). I have even been told that this is indicative of cardinal superiority. Following that idea, the Packs that are never emptied would indicate that $\mathbb{N}$ has a strictly higher cardinality than $P(\mathbb{N})$. :D.\\\\
	
	\noindent
	Negating A would mean:\\\\
	(NOT\_A.1): We cannot deny the final result of empty. This would mark an infimum value\footnote{A value that may not belong to the infinite series of values, but NOTHING greater than it is unattainable by the decreasing series of values} for any possible state of NWSP or The Packs, that doesn't even have a single element. Let us not forget that the states of NWSP and The Packs are nested sets that strictly contain each other. WE ARE NOT GOING TO IGNORE THIS.\\\\
	(NOT\_A.2): It is stupid to say that a set has a cardinal greater than 0 and be unable to name a single element within it. Not because we don't know it, but because we are absolutely certain that ALL its elements, at some point, cease to belong to the set. WE ARE NOT GOING TO IGNORE THIS.\\\\
	(NOT\_A.3): WE CAN IGNORE that the set we want to empty has a cardinality of $\infty$ in each term of the infinite intersection. As I say in the TPI document, both in the concept of limit and in the example of Achilles and the turtle, the idea of a distance that we can always reduce is used. And it is ignored, without any complex or problem, that every $\epsilon$ or $\delta$, less than any chosen distance, always differs from the true point under study by an infinite number of points. It is not something 'new' in mathematics.\\\\
	
	\noindent
	\textit{Even both infinite intersections, in both constructions, have only $\aleph_{0}$ terms.}\\\\

	\noindent
	I start by saying that it is possible to construct the equivalent of an impossible injective relation according to Cantor's Theorem:\\
	a) For A=true, constructing the DI is possible.\\
	b) For A=false, constructing the TPI is possible.\\
	
	\noindent
	The interpretation of the infinite intersection can be ambiguous. But for each possible interpretation, there exists an impossible injective relation according to Cantor's Theorem. ONE OF THE TWO MUST BE WELL CONSTRUCTED. Since in both, the only doubtful point is A. Therefore: ONE EXISTS. There is a counterexample to the theorem.\\\\
	
	\noindent
	They denied me the DI by saying that the empty set was not an infimum, but it was an EFFECTIVE result. My Packs emptied. It was impossible for me to have even ONE element, per Pack, with which to construct the unique image of each element in the Domain. None survived the infinite process of discarding from the DI (we will see this in another document). It was not possible to construct the injective relation. I didn't have a SINGLE ONE... but of course, if we apply the same idea to NWSP... if NWSP empties out effectively, that means that WSP is (Domain X Domain): an injective relation by Cantorian definition. If NWSP does not empty, I do not have an injective relation according to TPI (ignoring for now the option of the union of all disjoint subsets of the source set). But if NWSP does not empty, then the Packs don't either, and I would have unique options, within each Pack, to construct an injective relation. It's a dead end: a siege. All possibilities are covered. One cannot be denied without authorizing the other.\\\\
	
	\noindent
	Both results are contradictory ONLY if we consider $\aleph_{0}$ and $\aleph_{1}$ to be distinct cardinals. Exchanging the cardinal role between two sets with the same infinite cardinal is common. It is possible to construct relations that make one appear greater than the other, and vice versa.\\\\
	
	\noindent
	Let's consider the natural numbers ($\mathbb{N}$) and the even numbers ($\mathbb{P}$, chosen letter from its name in Spanish):\\\\
	$f_{1}: \mathbb{P} \longrightarrow \mathbb{N}$\\
	$f_{1}(p) = \{p, p+1\}$\\
	*\textit{We have two natural numbers for each even number.}\\\\
	$f_{2}: \mathbb{N} \longrightarrow \mathbb{P}$\\
	$f_{2}(n) = \{n * 10^{4}, (n * 10^{4})+2, (n * 10^{4})+4, ... , (n * 10^{4})+9998 \}$\\
	*\textit{We have 5000 even numbers for each natural number.}\\\\
	
	\noindent
	The DI and TPI are not contradictory, both can be constructed, PRECISELY, because $\aleph_{0}$ was ALWAYS equal to $\aleph_{1}$. That's why the roles of the sets involved in the infinite intersections can be exchanged.\\\\
	
	\noindent
	That's why it's necessary to add the DI, to complete the siege. Although I didn't expect it, the truth is that it was rejected as 'crankery' in its day and now it's being used to deny TPI. The simple example we're going to see below is THAT, a simple example. It shows a siege technique, which is to change the definition of ALL possible sets involved, to see that once changed, the problematic set does not generate any problem, being the same set.\\\\ 
	
	\noindent
	In the TPI, applied to $P(\mathbb{N})$ and $\mathbb{N}$, we will replace the subsets of $\mathbb{N}$ with $SNEFs$ and $SNEIs$. This simple change disables Cantor's proof in various ways, one of which is allowing the construction of a TPI. And as I have seen that things are denied without thinking twice, as long as the Theorem remains alive, even falling into circular arguments, where the theorem is used as a premise of the theorem itself... The DI is also necessary, not just the change in the definition of the sets. The TPI is so well constructed that the only possibility for Cantor's Theorem to remain alive is to resort to an old work OF MINE that was considered 'crankery' in its day. Living to see the arguments that two different mathematicians can come up with. Yes, I am disappointed. I warn them, even if they choose that path, and their solution is to cut off communications without facing their own judgments.\\\\
	
	\noindent
	I know you're going to get angry at the first glance of the simple example. I've gotten some very strange reactions. From accusing me of having chosen it 'too well', to attacking its simplicity. The first one is that I didn't knew mathematical generalizations could have exceptions. The second one is extremely serious: yes, it's too simple, and statistically, a couple of observations that are equally simple to see are going to be overlooked. So please read it twice... if you stop halfway because of that virulent reaction, remember to finish the first chapter, just the first... and you'll understand what I'm talking about. If you think I can't reproduce the phenomenon for more complex sets, you wouldn't actually have found a flaw, you'd just be taking a step back. I would have already demonstrated to you that the phenomenon is possible in one case... and opened up the question of whether there might be more than one. I understand your time is valuable, but I'm not coming empty-handed: I offer you a little crack in the theorem, simple, but a little crack. I'm telling you I can do it for more complex sets. I have references from a professor. I have mathematicians using my own work to deny ANOTHER one of my works (and vice versa). I have every point of the two constructions checked by, at least, two different people. Unofficially, because everyone's time is valuable, and I'm just a supposed 'crankery'. I HAVE DONE MY WORK. Are you going to do yours as guardians of mathematical knowledge? Even before burying yourselves in more text, I try to offer you things to show that the effort is worth it.\\\\
	
	\noindent
	That being said, and with the abstract READ. Let's begin.\\\\