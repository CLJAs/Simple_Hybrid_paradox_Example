\newpage
\renewcommand{\abstractname}{Abstract Español}
\begin{abstract}
	
	\noindent
	Esta serie de documentos no pretenden ir más allá de presentar hechos matemáticos. Poco frecuentes dada su naturaleza y su origen, pero hechos al fin y al cabo. Existe constancia abundante y frustrante sobre como provocan reacciones muy extrañas y virulentas, y no precisamente por ser erróneos. Un mismo punto que dispara en una persona una reacción virulenta de negación, otras lo consideran algo trivial y obvio. No son opiniones personales del escritor de este documento. Existen numerosas anécdotas que se podrían contar, pero es mejor experimentarlo, para tomar consciencia. Es extraño, pero ese proceso de consciencia forma parte de la prueba. Fenómenos lógicos que no se creen hasta que se observan por primera vez. Y se sufren.\\\\
	
	\noindent
	La primera lectura rápida del caso reducido que se va a proponer, está comprobado que provoca reacciones airadas. Pero es esa misma reacción la que impide la observación de cuestiones muy simples, que están a simple vista. Se ruega una doble lectura, aunque solo sea del primer capítulo no introductorio, y no dejarlo a medias, para poder comprobar por nosotros mismos como sucede.\\\\
	
	\noindent
	Este documento se añade, en la posición I.0.5, a la serie de equivalencias propuesta en:\\
	`Presentation of the Unlimited Transference of Pairs Method: an Alternative to Bijections'\footnote{https://vixra.org/abs/2209.0120}\\
	En su último capítulo.\\\\
	
	\noindent
	Los puntos que se tratarán serán:\\
	
	\noindent
	1) Presentación del ejemplo reducido de una paradoja híbrida y su desactivación mediante un `asedio' lógico. Ejemplo que crea una pequeña grieta en la demostración del Teorema de Cantor. Si existe un caso, ¿existirán más? Obviamente el asedio será más complejo para la totalidad de $P(\mathbb{N})$. De ahí la necesidad de varios documentos y una larga exposición. A pesar de ser reducido y muy simple, ya se podrán empezar a observar cuestiones importantes, que se espera que sirvan como invitación a realizar el esfuerzo de leer la serie entera.\\
	
	\noindent
	2) Explicar, en un capítulo introductorio, el por qué de la necesidad de añadir este documento, y uno más al final, para complementar la TPI\footnote{Transferencia de Pares Ilimitada}. La técnica de la DI\footnote{Diagonalización Inversa: variante pelea de colegio}. Admito haberme quedado estupefacto al observar, como sin saberlo, algunas personas proponen la DI como único argumento sólido para poder negar la TPI. La DI en su día fue rechazada como `crankery', y la TPI fue creada usando ese feedback. Se da el curioso fenómeno que son complementarias. No se puede negar una sin autorizar la otra.\\
	
	\noindent
	3) Un capítulo apéndice, para dejar constancia de referencias, comentar reacciones a la TPI y ya que se saca el tema de las paradojas híbridas, comentar mis hipótesis al respecto, para ofrecer pistas sobre caminos con los que ampliar estos trabajos. Optativo, solo para curiosos.
	
					
\end{abstract}