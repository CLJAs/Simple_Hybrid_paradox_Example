\newpage
\renewcommand{\abstractname}{Abstract Español}
\begin{abstract}
	
	\noindent
	No es orgullo, es determinación. No pretendo humillar a nadie, solo presentar realidades matemáticas. La gente cortocircuita al ver mis resultados. Puedo contar mil anécdotas o lo puedo enseñar. Podéis ser vosotros mismos sujetos del pequeño experimento que será el primer capítulo de este documento. Estadísticamente, la primera lectura, del primer capítulo no introductorio, os va a ocasionar una reacción muy virulenta. Una segunda lectura os hará daros cuenta de los detalles OBVIOS que se os han escapado por culpa de la primera reacción.\\
	
	\noindent
	Este documento pretende:\\\\
	1) Presentar una versión muy sencilla de lo que realmente sucede con la demostración del Teorema de Cantor. Podréis observar como funciona un `asedio' para detectar a una paradoja híbrida.\\\\
	2) ¿Si existe UN caso, existirán más? Obviamente el asedio será mucho más complejo para la totalidad de $P(\mathbb{N})$. De ahí la serie de documentos. Con el ejemplo sencillo ya podremos observar ciertas propiedades y sobre todo, reacciones personales.\\\\   
	3) Explicar en la introducción POR QUÉ he necesitado añadir DOS documentos nuevos a la serie inicialmente planeada. La Diagonalización Inversa (DI: variante pelea de colegio) y este. Ha sido muy frustrante observar como niegan la TPI\footnote{https://vixra.org/abs/2209.0120} usando una interpretación del mismo tipo de intersecciones infinitas, que yo ya propuse hace tres años, y que encima convierte en absolutamente correcta la DI. Básicamente me han dicho que debería recibir una educación formal matemática para entender mi propio trabajo rechazado hace tres años. Por cierto, rechazado usando la interpretación que uso en la TPI.\\\\
	4) Capítulos apéndice para dejar constancia de mis hipótesis, y respuestas a la TPI. Complementarios, solo para curiosos.
	
				
\end{abstract}