\newpage
\renewcommand{\abstractname}{Abstract English}
\begin{abstract}
	
	\noindent
	This series of documents do not intend to go beyond presenting mathematical facts. Uncommon due to their nature and origin, but facts nonetheless. There is abundant and frustrating evidence about how they cause very strange and virulent reactions, not precisely because they are wrong. The same point that triggers a virulent reaction of denial in one person, others consider it something trivial and obvious. These are not personal opinions of the writer of this document. There are numerous anecdotes that could be told, but it is better to experience it to become aware. It is strange, but that process of awareness is part of the proof. Logical phenomena that are not believed until they are observed for the first time. And suffered.\\\\
	
	\noindent
	The first quick reading of the reduced case that is going to be proposed has been proven to provoke angry reactions. But it is that same reaction that prevents the observation of very simple issues that are in plain sight. A double reading is requested, even if it is only of the first non-introductory chapter, and not leaving it halfway, in order to be able to verify for ourselves how it happens.\\\\
	
	\noindent
	This document is added, in position I.0.5, to the series of equivalences proposed in:\\
	`Presentation of the Unlimited Transference of Pairs Method: an Alternative to Bijections'\footnote{https://vixra.org/abs/2209.0120}\\
	In its last chapter.\\\\
	
	\noindent
	The points that will be addressed are:\\
	
	\noindent
	1) Presentation of the reduced example of a hybrid paradox and its deactivation through a logical 'siege'. An example that creates a small crack in the proof of Cantor's Theorem. If there is one case, will there be more? Obviously, the siege will be more complex for the entirety of $P(\mathbb{N})$. Hence the need for several documents and a long exposition. Despite being reduced and very simple, important issues can already be observed, which are expected to serve as an invitation to make the effort to read the entire series.\\
	
	\noindent
	2) Explain, in an introductory chapter, the reason for the need to add this document, and one more at the end, to complement the TPI\footnote{Letters in spanish, meaning Unlimited Transfer of Pairs}. The technique of DI\footnote{Inverse Diagonalization: schoolyard fight variant}. I admit to being stunned to observe how, without knowing it, some people propose DI as the only solid argument to deny TPI. DI was rejected in its day as 'crankery', and TPI was created using that feedback. The curious phenomenon is that they are complementary. One cannot be denied without authorizing the other.\\
	
	\noindent
	3) An appendix chapter, to record references, comment on reactions to TPI and, since the subject of hybrid paradoxes is raised, to comment on my hypotheses in this regard, to offer clues about ways to expand these works. Optional, only for the curious.

\end{abstract}