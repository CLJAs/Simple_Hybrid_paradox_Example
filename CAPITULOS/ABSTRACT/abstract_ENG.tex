\newpage
\renewcommand{\abstractname}{Abstract English}
\begin{abstract}
	
	\noindent
	It's not pride, it's determination. I don't intend to humiliate anyone, just present mathematical realities. People short-circuit when they see my results. I can tell a thousand anecdotes or I can teach you. You can be the subjects of the small experiment that will be the first chapter of this document. Statistically, the first reading of the first non-introductory chapter will cause a very virulent reaction. A second reading will make you realize the OBVIOUS details that have escaped you due to the first reaction.\\
	
	\noindent
	This document aims to:\\\\
	1) Present a very simple version of what really happens with the demonstration of Cantor's Theorem. You will be able to observe how a `siege' works to detect a hybrid paradox.\\\\
	2) If there is ONE case, will there be more? Obviously, the siege will be much more complex for the total of $P(\mathbb{N})$. Hence the series of documents. With the simple example, we will already be able to observe certain properties and, above all, personal reactions.\\\\
	3) Explain in the introduction WHY I have needed to add TWO new documents to the initially planned series. The Inverse Diagonalization (DI: school fight variant) and this one. It has been very frustrating to observe how they deny TPI\footnote{https://vixra.org/abs/2209.0120} using an interpretation of the same type of infinite intersections, which I already proposed three years ago, and which even makes DI absolutely correct. Basically, they told me I should receive a formal mathematical education to understand my own rejected work three years ago. By the way, rejected using the interpretation I use in TPI.\\\\
	4) Appendix chapters to keep a record of my hypotheses, and answers to TPI. Complementary, just for the curious.

\end{abstract}