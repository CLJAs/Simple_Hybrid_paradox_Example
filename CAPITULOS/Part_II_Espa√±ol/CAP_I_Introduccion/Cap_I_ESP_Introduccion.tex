\chapter{Introducción}
	
	\noindent
	Lo que interesa de este documento es mostrar el ejemplo reducido de paradoja híbrida, y explicar la relación complementaria entre la DI y la TPI. Es aceptable saltarse este capítulo de introducción e ir al siguiente para ver el ejemplo reducido si no se tiene curiosidad por la explicación de la necesidad de añadir la DI. Es interesante porque ayuda a entender el fallo de la mayor crítica a la TPI, pero no deja de ser un adelanto de lo que se verá al final de toda la serie.\\\\
	
	\noindent 
	Tanto la DI como la TPI dependen de la interpretación de intersecciones infinitas muy similares, casi idénticas, en circunstancias y propiedades. Las dos construcciones, si estuviesen bien construidas, serían equivalentes a una relación inyectiva imposible entre $P(\mathbb{N})$ y $\mathbb{N}$. Tanto la TPI como la DI, necesitan la propiedad `A', pero de formas complementarias. Una triada de interpretaciones.\\\\
	
	\noindent
	\textbf{Afirmar A consistiría en:}\\\\
	(A.1): Podemos ignorar el resultado final de vacío. Digo ignorar porque no se puede negar. En ambos casos la intersección infinita tiene un resultado INNEGABLE de vacío.\\\\
	(A.2): Hay un conjunto que puede ser vacío o no: el 'conjunto rotativo´. Somos incapaces de mencionar un solo elemento dentro del conjunto, porque sabemos con absoluta seguridad, que CADA UNO de sus elementos es eliminado de él, a partir de un paso concreto y calculable de la intersección infinita. El hecho de no ser capaces de mencionar un solo elemento dentro de él, LO VAMOS A IGNORAR TAMBIÉN... argumentando la observación A.3.\\\\
	(A.3): El 'conjunto rotativo´es NWSP en el caso de la TPI, y los Packs\footnote{Ya veremos que son los Packs en futuros documentos} de naturales únicos en la DI. Siempre tiene un cardinal infinito en cada paso, de los infinitos términos de la intersección infinita. Esto es innegable también. Pero decidimos NO IGNORAR esto, para decir que ningún subconjunto (subconjuntos del conjunto Origen en la TPI, familias de pares del Dominio en la DI) nos obliga jamás a vaciar el conjunto objetivo (NWSP o Los Packs). Algunas personas han llegado a afirmar que eso es indicativo de superioridad cardinal. Siguiendo esa idea, los Packs que nunca se vacían indicarían que $\mathbb{N}$ tiene un cardinal estrictamente superior al de $P(\mathbb{N})$.\\\\
	
	\noindent
	\textbf{Negar A, significaría:}\\\\
	(NOT\_A.1): No podemos negar el resultado final de vacío. Esto marcaría un valor infimum\footnote{Un valor que igual no pertenece a la serie infinita de valores, pero NADA mayor que él, es inalcanzable por la serie de valores o estados decrecientes} para todo posible estado de NWSP, o los Packs, que ni siquiera tiene un solo elemento. No olvidemos que los estados de NWSP, y los Packs, son conjuntos anidados que se contienen estrictamente los unos a los otros. NO LO VAMOS A IGNORAR.\\\\
	(NOT\_A.2): Es estúpido decir que un conjunto tiene un cardinal mayor que $0$, y ser incapaz de nombrar un sólo elemento dentro de él. No porque lo desconozcamos, sino porque estamos absolutamente seguros que TODOS sus elementos, en algún momento, dejan de pertenecer al conjunto. NO LO VAMOS A IGNORAR.\\\\
	(NOT\_A.3): PODEMOS IGNORAR que el conjunto que queremos vaciar, tenga cardinal $\infty$ en cada término de la intersección infinita. Como digo en el documento de la TPI, tanto en el concepto de límite, como en el ejemplo de Aquiles y la tortuga, se usa la idea de una distancia que siempre podemos reducir. Y se ignora, sin ningún complejo o problema, que cada $\varepsilon$ o $\delta$, menor a cualquier distancia escogida, siempre dista del verdadero punto a estudiar una cantidad infinita de puntos. No es algo 'nuevo´ en matemáticas.\\\\
	
	\noindent
	La DI fue rechazada en su día diciendo que el vacío no era un infimum, sino que era un resultado EFECTIVO. Mis Packs se vaciaban. Me resultaba imposible tener UN SOLO elemento, dentro de cada Pack, con el que construir la imagen única de cada elemento del Dominio. No sobrevivía NINGUNO al infinito proceso de descarte de la DI (ya lo veremos en otro documento). No era posible construir la relación inyectiva. No tenía NI UNO SOLO... pero claro, si aplicamos la misma idea a NWSP... si NWSP se vacía de forma efectiva, eso significa que WSP es (Dominio X Dominio): una relación inyectiva por definición cantoriana, o que cumple el Naive CA Theorem\footnote{Una condición equivalente a la inyectividad para relaciones que no son función}. Ahora me dicen que NWSP nunca se vacía. Si NWSP no se vacía, no tengo una relación inyectiva según la TPI (ignorando la opción de la unión de todos los subconjuntos disjuntos del conjunto origen, o que sigue funcionando aún, solo con la cercanía al vacío). Si NWSP no se vacía, los Packs tampoco, y tendría opciones únicas, dentro de cada Pack, para construir una relación inyectiva. Es un callejón sin salida: un asedio. Todas las posibilidades están cubiertas. No se puede negar una sin autorizar la otra.\\\\
	
	\noindent
	\textit{Incluso ambas intersecciones infinitas, en ambas construcciones, tienen solo $\aleph_{0}$ términos.}\\\\
	
	\noindent
	Parto de decir que es posible construir el equivalente a una relación inyectiva imposible, según el Teorema de Cantor:\\
	a) Para A=verdadero, construir la DI es posible\\
	b) Para A=falso, construir la TPI es posible\\\\
	
	\noindent
	La interpretación de la intersección infinita puede ser ambigua. Pero para cada posible interpretación, existe una relación inyectiva imposible según el Teorema de Cantor. UNA DE LAS DOS DEBE ESTAR BIEN CONSTRUIDA. Ya que en ambas, el único punto dudoso es A. Por lo tanto: UNA EXISTE. Existe UN contraejemplo del Teorema de Cantor.\\\\
	
	\noindent
	El asedio en el caso reducido consiste, al igual que en la TPI, en cambiar la definición de cada posible conjunto implicado. Crear definiciones alternativas que generan EXACTAMENTE los mismos conjuntos. Trabajar con ellas para observar como se desactiva la demostración del Teorema y como es posible generar cosas que se suponían imposibles. Los matemáticos suelen afirmar cualquier cosa con tal de que el teorema siga vivo. Han llegado a afirmar lo mismo y lo contrario con absoluta seguridad. No meditan mucho las consecuencias. La TPI plantea tan bien las cosas, que negar su única debilidad lleva a decir que $\mathbb{N}$ tiene un cardinal estrictamente mayor que $P(\mathbb{N})$. Incluso a defender el Teorema con argumentos de un trabajo que fue rechazado hace tres años. Para cortar de raíz esa tentación, es indispensable añadir la DI a la serie de documentos. El asedio es más sólido.\\\\
	
	\noindent
	Ambas construcciones son contradictorias SOLO si consideramos que $\aleph_{0}$ y $\aleph_{1}$ son cardinales distintos. Intercambiar el papel cardinal entre dos conjuntos con el mismo cardinal infinito, es algo común. Es posible construir relaciones que hagan a uno aparentemente mayor que el otro, y viceversa.\\\\
	
	\noindent
	Consideremos los números naturales ($\mathbb{N}$) y los números pares ($\mathbb{P}$, letra escogida de su nombre en español):\\
	**\textit{No son funciones, son relaciones. Cada elemento del Dominio tiene diferentes imágenes que se representan por el conjunto resultado. Nos podemos fijar en que los conjuntos Imagen, de cada elemento del Dominio, son disjuntos todos entre sí.}\\\\
	$f_{1}: \mathbb{P} \longrightarrow \mathbb{N}$\\
	$f_{1}(p) = \{p, p+1\}$\\
	*\textit{Tenemos dos números naturales por cada número par.}\\\\
	$f_{2}: \mathbb{N} \longrightarrow \mathbb{P}$\\
	$f_{2}(n) = \{n*10^{4}, (n*10^{4})+2, (n*10^{4})+4,(n*10^{4})+6, ... , (n*10^{4})+9998 \}$\\
	*\textit{Tenemos 5000 números pares por cada número natural.}\\\\
	
	\noindent
	El apéndice solo se añade por dejar constancia. Son cosas que no están chequeadas por falta de recursos. Las pongo por tener todo ordenado ya en una sola serie de publicaciones. Cuando se acaben, continuarán con las Construcciones LJA y sus técnicas avanzadas. Se escriben como pistas para futuros trabajos, como por ejemplo, el de tratar de enumerar TODOS los ordinales, o ir más allá de cosas que he visto en foros como $\aleph_{w}$, que es asequible para una CLJA. Hipótesis, cartas de referencias de catedráticos, comentarios sobre documentos anónimos que me han enviado... Nada indispensable para el argumento central, pero si se tiene curiosidad pueden ayudar a responder algunas preguntas, como por ejemplo: ¿Qué sucede con todas las demostraciones del Teorema? ¿Qué pistas podemos tener sobre los fallos de la demostración de Cantor? Procuro ser diplomático, porque no niego que el viaje está siendo frustrante. Por si acaso, me permito recordar que la existencia de un solo contraejemplo invalida cualquier demostración. Tengo pistas muy buenas sobre la demostración de Cantor, y algunas sobre otras demostraciones, pero necesitaría ayuda para desarrollarlas.\\\\
	
	\noindent
	En el siguiente episodio podremos empezar a hacernos una idea de lo que está sucediendo. Si se reacciona como otras personas lo han hecho en el pasado, pero aún así se le concede al siguiente capítulo, una segunda lectura, es posible darse cuenta de lo sencillo que era caer en esos errores. De lo sutiles que son, y de la extraña sensación que provocan en la cabeza. Las paradojas híbridas son fenómenos lógicos muy escurridizos. Aunque pueda parecer un ejemplo simple, `atraparla' dentro, con ese diseño, ha costado años de ensayo y error, aún sabiendo desde el inicio por donde iban los tiros.\\\\
	
	\noindent
	En realidad, desmontar el Teorema de Cantor es sólo una herramienta para demostrar que las paradojas híbridas existen, y que llevan a teoremas falsos. Pero definirlas bien y aprender a detectarlas mejor es harina de otro costal. A mi me ha costado más de 25 años `atrapar' una. El siguiente ejemplo es solo un adelanto. Pero no deja de ser una versión reducida de todo lo que va a provocar el ejemplo aplicado a la totalidad de $P(\mathbb{N})$.\\\\ 


