\chapter{Introducción}

	\noindent
	¿Has leído el abstract? Asegúrate primero de haberlo hecho. Luego la gente me dice que no entiende a donde quiero llegar. Este documento se ha creado para responder a las reacciones de la TPI... y de paso daros algo que comience a aportar datos sobre la hipótesis de la existencia de las paradojas híbridas. NO ES EL CONTRAEJEMPLO. Solo es un ejemplo sencillo para daros ánimos a leer uno mucho más complejo y largo.\\\\
	
	\noindent
	No puedo evitarlo, escribo un poco desordenado. No sé como resolver el problema de que creáis que lo que enseño en cada documento, es lo ÚNICO que tengo, y si trato de advertirlo me decís que soy desordenado. Ni siquiera esta serie de documentos es TODO lo que tengo. El documento de la TPI tiene 60 páginas. Intenté ahorrármelas con una persona de internet, alguien que no podía leer español, presentando solo lo que es una `inyección imperfecta', sin ejemplos, y la definición de la TPI. El resultado fue que hizo TODO lo que suplico que no se haga en su introducción. Pondré el documento adjunto que me mandó `anónimo' en los capítulos apéndices. En el no se pueden ver nuestras conversaciones, donde habla claramente de `conjuntos rotativos' y me indica que YO sería capaz de verlo si hubiese tenido una educación formal que me quitase todas estas tonterías de la cabeza. Pero es que los `conjuntos rotativos' son mi propio argumento en la DI. El que me negaron y por eso construí la TPI.\\\\
	
	\noindent
	Recordemos que NO soy matemático. Aún así tengo una carta de un catedrático\footnote{Texto en el capítulo apéndice} diciendo que ha sido incapaz de encontrar un fallo. Si el formato o el rigor, fuesen un fallo REAL, más allá de la rabieta inicial que os puede provocar, lo hubiese mencionado. Bueno, de hecho... lo menciona... y aún así no lo considera un fallo grave. PRECISAMENTE estoy harto de pedir ayuda para poder escribir todo mi material, de una forma más convencional, y poder chequearlo de una forma más profunda y expandirlo. Ante esa ausencia de recursos y personas cualificadas, lo único que nos queda es el imperativo científico, filosófico y moral, que una verdad es más valiosa que una mentira con una presentación cómoda. Al menos, en matemáticas, espero. Yo he ido más allá de dar lo mejor de mi mismo.\\\\
	
	\noindent
	No paro de advertir que delante de una pizarra es más ameno, porque me puedo saltar suposiciones innecesarias, y adaptarme, estudiando y evaluando el feedback directo de cada persona. \\\\
	
	\noindent
	Cada idea, cada concepto, ha sido defendido en foros, y tras varios feedbacks y correcciones, han sido chequeados y considerados correctos. Cada resultado también. Excepto el punto que obliga a añadir la DI, la interpretación de las intersecciones infinitas, lo cual consideraba innecesario después de que me tumbasen el concepto de `conjuntos rotativos'. Ya lo veremos en documentos futuros, o puede que tu ya hayas empezado a fraguarlo en tu mente, para poder negar la TPI, si has leído el documento COMPLETO. Me harto de decirlo pero parece ser que no se escucha (lee con cuidado): si DECIDES que los subconjuntos de la partición del CONJUNTO ORIGEN en la TPI, jamás llegan ni a estar cerca del cardinal del conjunto Dominio, eso indica que crees firmemente que existe UN conjunto diferente del vacío, restando a Dominio X Dominio, el mayor subconjunto (NWSP) posible, o TODOS ellos ( recordemos que son anidados ). Después de haberlo intentando con TODOS los subconjuntos disjuntos del conjunto Origen en cada relación. A pesar de que serás incapaz de mencionar un solo elemento dentro de él. Eso te llevará a IGNORAR el INNEGABLE resultado final de vacío, y a apoyarte en que para cada subconjunto siempre nos dejamos sin abarcar un conjunto infinito de elementos de NWSP. Una triada de decisiones inseparables, como he podido comprobar en las reacciones de varias personas. Si ese conjunto `sobrante' existe de alguna forma extraña, podemos escoger uno o más elementos de él... y en eso se basa la DI (que ya veremos en documentos futuros). En la DI al que le cuesta `eliminar' todos los elementos es al conjunto con supuesto cardinal $\aleph_{1}$. En la DI decidimos ignorar el resultado final de vacío de la intersección, que se produce en los mismos términos pero con los papeles invertidos, y solo nos fijamos en la impotencia del conjunto con cardinal $\aleph_{1}$ para deshacerse de un simple conjunto con cardinal $\aleph_{0}$.\\\\
	
	\noindent
	Si no te vas a molestar en contactarme para preguntarme dudas, o lo vas a leer rápido, saltándote partes\footnote{LÉETE EL ABSTRACT!!}, etc... Por favor, ni lo intentes. YA SABES que tengo defectos como escritor, cabrearse por ello, en vez de enviarme un e-mail preguntándome... no sé como calificarlo. Leerlo rápido te va a llevar a plantearte auténticas tonterías (demasiadas experiencias previas)... Una persona me subrayó casi un párrafo entero diciéndome que una propiedad concreta no tenía por qué funcionar en conjuntos infinitos. La frase que se dejó sin subrayar, justo al inicio del mismo párrafo, decía algo así como: "...esto aplica solo a conjuntos de cardinalidad finita". PREGUNTA, lee con cierto cariño y paciencia, y si te cabreas, recuerda que tengo dobles chequeos, gente potente que no encuentra el fallo y que me he ofrecido a hacerlo en persona miles de veces con una pizarra, y me llaman prepotente por ello. Esa duda, o apreciación de error, la podría resolver en SEGUNDOS, y hay más de dos matemáticos que NO lo consideran un error. Sobre todo con la presencia de más matemátic@s, podríamos dejar los argumentos de autoridad a un lado, ya que podrías ver sus caras, no la mía. NO ME IMPORTA RESPONDER UN MILLÓN DE VECES, yo soy el primero que no es matemático aquí. Lo que me molesta es que se corte la comunicación por auténticas tonterías. A veces con argumentos indignos de un matemático. Cosas como decir que un contraejemplo no anula un teorema, ya que se debe explicar, aparte, el POR QUÉ falla cada una de las demostraciones que existen. En fin...\\\\
	
	\noindent
	Esta semana he podido chequear algo que me producía una frustración brutal. No creía que fuese un argumento tan complejo de entender, pero 'alguien´ ya me ha confirmado que es correcto. Incluso esa persona no entendía por qué insistía en la pregunta, y me repitió varias veces que era correcto. Como siempre, trabajando con gente anónima en foros, con la que no puedo seguir trabajando cuando cortan la comunicación, y pasan meses hasta que vuelvo a encontrar a otra persona.\\\\
	
	\noindent
	La idea es simple, las dos construcciones, si estuviesen bien construidas, serían equivalentes a una relación inyectiva imposible entre $P(\mathbb{N})$ y $\mathbb{N}$. Tanto la TPI (Transferencia de Pares Ilimitada) como la DI (Diagonalización Inversa), necesitan la propiedad A, pero de formas complementarias. Una triada de interpretaciones sobre el tipo de intersecciones infinitas que se presentan en el documento de la TPI. Afirmar A consistiría en:\\\\
	(A.1): Podemos ignorar el resultado final de vacío. Digo ignorar porque no se puede negar. En ambos casos la intersección infinita tiene un resultado INNEGABLE de vacío.\\\\
	(A.2): Hay un conjunto que puede ser vacío o no: el 'conjunto rotativo´. Somos incapaces de mencionar un solo elemento dentro del conjunto, porque sabemos con absoluta seguridad, que CADA UNO de sus elementos es eliminado de él, a partir de un paso concreto y calculable de la intersección infinita. El hecho de no ser capaces de mencionar un solo elemento dentro de él, LO VAMOS A IGNORAR TAMBIÉN... argumentando la observación A.3.\\\\
	(A.3): El 'conjunto rotativo´, NWSP en el caso de la TPI, los Packs\footnote{Parece una tontería recordarlo, pero dadas experiencias previas: ya veremos que son los Packs en futuros documentos} de naturales únicos en la DI, siempre tiene un cardinal infinito en cada paso, de los infinitos términos de la intersección infinita. Esto es innegable también. Pero decidimos NO IGNORAR esto, para decir que ningún subconjunto (subconjuntos del conjunto Origen en la TPI, familias de pares del Dominio en la DI) nos obliga jamás a vaciar el conjunto objetivo (NWSP o Los Packs). Incluso me han llegado a afirmar que eso es indicativo de superioridad cardinal. Siguiendo esa idea, los Packs que nunca se vacían indicarían que $\mathbb{N}$ tiene un cardinal estrictamente superior al de $P(\mathbb{N})$.:D.\\\\
	
	\noindent
	Negar A, significaría:\\\\
	(NOT\_A.1): No podemos negar el resultado final de vacío. Esto marcaría un valor infimum\footnote{Un valor que igual no pertenece a la serie infinita de valores, pero NADA mayor que él, es inalcanzable por la serie de valores decrecientes} para todo posible estado de NWSP, o los Packs, que ni siquiera tiene un solo elemento. No olvidemos que los estados de NWSP, y los Packs, son conjuntos anidados que se contienen estrictamente los unos a los otros. NO LO VAMOS A IGNORAR.\\\\
	(NOT\_A.2): Es estúpido decir que un conjunto tiene un cardinal mayor que $0$, y ser incapaz de nombrar un sólo elemento dentro de él. No porque lo desconozcamos, sino porque estamos absolutamente seguros que TODOS sus elementos, en algún momento, dejan de pertenecer al conjunto. NO LO VAMOS A IGNORAR.\\\\
	(NOT\_A.3): PODEMOS IGNORAR que el conjunto que queremos vaciar, tenga cardinal $\infty$ en cada término de la intersección infinita. Como digo en el documento de la TPI, tanto en el concepto de límite, como en el ejemplo de Aquiles y la tortuga, se usa la idea de una distancia que siempre podemos reducir. Y se ignora, sin ningún complejo o problema, que cada $\epsilon$ o $\delta$, menor a cualquier distancia escogida, siempre dista del verdadero punto a estudiar una cantidad infinita de puntos. No es algo 'nuevo´ en matemáticas.\\\\
	
	\noindent
	\textit{Incluso ambas intersecciones infinitas, en ambas construcciones, tienen solo $\aleph_{0}$ términos.}\\\\
	
	\noindent
	Parto de decir que es posible construir el equivalente a una relación inyectiva imposible, según el Teorema de Cantor:\\
	a) Para A=verdadero, construir la DI es posible\\
	b) Para A=falso, construir la TPI es posible\\\\
	
	\noindent
	La interpretación de la intersección infinita puede ser ambigua. Pero para cada posible interpretación, existe una relación inyectiva imposible según el Teorema de Cantor. UNA DE LAS DOS DEBE ESTAR BIEN CONSTRUIDA. Ya que en ambas, el único punto dudoso es A. Por lo tanto: UNA EXISTE. Existe UN contraejemplo del Teorema.\\\\
	
	\noindent
	La DI me la negaron diciendo que el vacío no era un infimum, sino que era un resultado EFECTIVO. Mis Packs se vaciaban. Me resultaba imposible tener UN SOLO elemento, por Pack, con el que construir la imagen única de cada elemento del Dominio. No sobrevivía NINGUNO al infinito proceso de descarte de la DI (ya lo veremos en otro documento). No era posible construir la relación inyectiva. No tenía NI UNO SOLO... pero claro, si aplicamos la misma idea a NWSP... si NWSP se vacía de forma efectiva, eso significa que WSP es (Dominio X Dominio): una relación inyectiva por definición cantoriana. Si NWSP no se vacía, no tengo una relación inyectiva según la TPI (ignorando por ahora la opción de la unión de todos los subconjuntos disjuntos del conjunto origen). Pero si NWSP no se vacía, los Packs tampoco, y tendría opciones únicas, dentro de cada Pack, para construir una relación inyectiva. Es un callejón sin salida: un asedio. Todas las posibilidades están cubiertas. No se puede negar una sin autorizar la otra.\\\\
	
	\noindent
	Ambos resultados son contradictorios SOLO si consideramos que $\aleph_{0}$ y $\aleph_{1}$ son cardinales distintos. Intercambiar el papel cardinal entre dos conjuntos con el mismo cardinal infinito, es algo común. Es posible construir relaciones que hagan a uno aparentar ser mayor que el otro, y viceversa.\\\\
	
	\noindent
	Consideremos los números naturales ($\mathbb{N}$) y los números pares ($\mathbb{P}$, letra escogida de su nombre en español):\\\\
	$f_{1}: \mathbb{P} \longrightarrow \mathbb{N}$\\
	$f_{1}(p) = \{p, p+1\}$\\
	*\textit{Tenemos dos números naturales por cada número par.}\\\\
	$f_{2}: \mathbb{N} \longrightarrow \mathbb{P}$\\
	$f_{2}(n) = \{n*10^{4}, (n*10^{4})+2, (n*10^{4})+4, ... , (n*10^{4})+9998 \}$\\
	* \textit{Tenemos 5000 números pares por cada número natural.}\\\\
	
	\noindent
	La DI y la TPI no son contradictorias, las dos se pueden construir, PRECISAMENTE, porque $\aleph_{0}$ SIEMPRE fue igual a $\aleph_{1}$. Por eso se pueden intercambiar los papeles de los conjuntos implicados en las intersecciones infinitas.\\\\
	
	\noindent
	Por eso es necesario añadir la DI, para completar el asedio. Aunque no me lo esperaba, la verdad. En su dia fue rechazada por 'crankery´ y ahora la usan para poder negar la TPI. El ejemplo sencillo que vamos a ver a continuación es ESO, un ejemplo sencillo. Muestra una técnica de asedio, que es cambiar la definición de TODOS los posibles conjuntos implicados, para ver que una vez cambiadas, el conjunto problemático no genera ningún problema, siendo el mismo conjunto. En la TPI, aplicada a $P(\mathbb{N})$ y $\mathbb{N}$, cambiaremos los subconjuntos de $\mathbb{N}$ por $SNEFs$ y $SNEIs$. Ese simple cambio deshabilita la demostración de Cantor, de varias formas, una de ellas es permitir construir una TPI. Y como he podido comprobar que las cosas se niegan sin pensar dos veces, con tal de que el Teorema siga vivo, incluso cayendo en argumentos circulares, donde se usa el teorema como premisa del propio teorema... La DI es necesaria también, no solo el cambio de definición de los conjuntos. La TPI está tan bien construida, que la única posibilidad de que el Teorema de Cantor siga vivo es tirar de un antiguo trabajo MIO que fue considerado 'crankery´ en su dia. Vivir para ver los argumentos que pueden llegar a usar DOS matemáticos diferentes. Si, estoy defraudado. Les aviso, aún así escogen ese camino, y su solución es cortar comunicaciones sin enfrentarse a sus propios juicios.\\\\
	
	\noindent
	Sé que te vas a cabrear en el primer vistazo al ejemplo simple. He obtenido reacciones muy extrañas. Desde acusarme de haberlo escogido 'demasiado bien´, hasta atacar su simpleza. Lo primero es que no sabía que las generalizaciones matemáticas podían tener excepciones. Lo segundo es sangrante: sí, es demasiado sencillo, y estadísticamente, se te van a pasar por alto un par de observaciones igualmente simples de ver. Así que lee dos veces por favor... si te paras a medias, por esa reacción virulenta, recuerda acabar el primer capítulo, solo el primero... y entenderás de lo que hablo. Si te piensas que no puedo reproducir el fenómeno para conjuntos más complejos, en realidad no habrías encontrado un fallo, sólo estarías dando un paso atrás. Ya te habría demostrado que el fenómeno es posible en un caso... y abierto la duda sobre si es posible que exista más de uno. Entiendo que tu tiempo es oro, pero no vengo con las manos vacías: te ofrezco una grieta en el teorema, simple, pero una grieta. Te digo que lo puedo hacer para conjuntos más complejos. Tengo referencias de un catedrático. Tengo matemáticos usando mi propio trabajo para negar OTRO de mis trabajos (y viceversa). Tengo cada punto de las dos construcciones, chequeado por, al menos, dos personas diferentes. De forma no oficial, porque el tiempo de todo el mundo es oro, y yo solo soy un supuesto 'crankery´. YO HE HECHO MI TRABAJO. ¿Vais a hacer el vuestro como guardianes del conocimiento matemático? Incluso antes de enterraros en más texto, procuro ofreceros cosas para demostrar que el esfuerzo merece la pena.
	\\\\ 
	
	\noindent
	Dicho esto, y con el abstract LEÍDO. Comencemos.
	
	




