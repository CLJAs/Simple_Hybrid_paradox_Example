\documentclass[notitlepage]{report}
\usepackage[left=1in, right=1in, top=1in, bottom=1in]{geometry}

%Para usar colores en las tablas:
\usepackage{color}
%\usepackage{graphicx} DUPLICADO
\usepackage{epsfig}
\usepackage{multirow}
\usepackage{colortbl}
\usepackage[table]{xcolor}
%Fin de pacquetes para usar colores en las tablas


\usepackage{titling}
\usepackage{lipsum}
\usepackage{mathtools}
\usepackage{amsmath}
\usepackage{amsfonts}
\usepackage{amssymb}
\usepackage{pdfpages}
\usepackage[spanish]{babel}
\usepackage[utf8]{inputenc}
\setlength{\parskip}{2mm}
\usepackage{graphicx}
\graphicspath{ {../Imagenes/Editadas/} } 

%Definiendo colores:
\definecolor{lightgray}{gray}{0.9}
\definecolor{myblue}{RGB}{180,241,231}
\definecolor{myred}{RGB}{241,121,108}
\definecolor{myyellow}{RGB}{245,239,122}
%Fin de definicion de colores:


\pretitle{\begin{center}\Huge\bfseries}
	\posttitle{\par\end{center}\vskip 0.5em}
\preauthor{\begin{center}\Large\ttfamily}
	\postauthor{\end{center}}
\predate{\par\large\centering}
\postdate{\par}

\title{P.I.0.5 Simple Hybrid Paradox Example.\\`The siege' and how is related to TPI \\ \Large{English/Español}}
\author{Juan Carlos Caso Alonso \& Francisco Mario Cruz Almeida}
%\date{\today}
\date{19 January/Enero 2023}

%\renewcommand{\chaptername}{C.-}
%\usepackage{fancyhdr}

\addto\captionsspanish{\renewcommand{\chaptername}{C.-}}

\begin{document}
	\maketitle
	\thispagestyle{empty}
	
	\newpage
\renewcommand{\abstractname}{Abstract English}
\begin{abstract}
	
	\noindent
	It's not pride, it's determination. I don't intend to humiliate anyone, just present mathematical realities. People short-circuit when they see my results. I can tell a thousand anecdotes or I can teach you. You can be the subjects of the small experiment that will be the first chapter of this document. Statistically, the first reading of the first non-introductory chapter will cause a very virulent reaction. A second reading will make you realize the OBVIOUS details that have escaped you due to the first reaction.\\
	
	\noindent
	This document aims to:\\\\
	1) Present a very simple version of what really happens with the demonstration of Cantor's Theorem. You will be able to observe how a `siege' works to detect a hybrid paradox.\\\\
	2) If there is ONE case, will there be more? Obviously, the siege will be much more complex for the total of $P(\mathbb{N})$. Hence the series of documents. With the simple example, we will already be able to observe certain properties and, above all, personal reactions.\\\\
	3) Explain in the introduction WHY I have needed to add TWO new documents to the initially planned series. The Inverse Diagonalization (DI: school fight variant) and this one. It has been very frustrating to observe how they deny TPI\footnote{https://vixra.org/abs/2209.0120} using an interpretation of the same type of infinite intersections, which I already proposed three years ago, and which even makes DI absolutely correct. Basically, they told me I should receive a formal mathematical education to understand my own rejected work three years ago. By the way, rejected using the interpretation I use in TPI.\\\\
	4) Appendix chapters to keep a record of my hypotheses, and answers to TPI. Complementary, just for the curious.

\end{abstract}
	\newpage
\renewcommand{\abstractname}{Abstract Español}
\begin{abstract}
	
	\noindent
	Esta serie de documentos no pretenden ir más allá de presentar hechos matemáticos. Poco frecuentes dada su naturaleza y su origen, pero hechos al fin y al cabo. Existe constancia abundante y frustrante sobre como provocan reacciones muy extrañas y virulentas, y no precisamente por ser erróneos. Un mismo punto que dispara en una persona una reacción virulenta de negación, otras lo consideran algo trivial y obvio. No son opiniones personales del escritor de este documento. Existen numerosas anécdotas que se podrían contar, pero es mejor experimentarlo, para tomar consciencia. Es extraño, pero ese proceso de consciencia forma parte de la prueba. Fenómenos lógicos que no se creen hasta que se observan por primera vez. Y se sufren.\\\\
	
	\noindent
	La primera lectura rápida del caso reducido que se va a proponer, está comprobado que provoca reacciones airadas. Pero es esa misma reacción la que impide la observación de cuestiones muy simples, que están a simple vista. Se ruega una doble lectura, aunque solo sea del primer capítulo no introductorio, y no dejarlo a medias, para poder comprobar por nosotros mismos como sucede.\\\\
	
	\noindent
	Este documento se añade, en la posición I.0.5, a la serie de equivalencias propuesta en:\\
	`Presentation of the Unlimited Transference of Pairs Method: an Alternative to Bijections'\footnote{https://vixra.org/abs/2209.0120}\\
	En su último capítulo.\\\\
	
	\noindent
	Los puntos que se tratarán serán:\\
	
	\noindent
	1) Presentación del ejemplo reducido de una paradoja híbrida y su desactivación mediante un `asedio' lógico. Ejemplo que crea una pequeña grieta en la demostración del Teorema de Cantor. Si existe un caso, ¿existirán más? Obviamente el asedio será más complejo para la totalidad de $P(\mathbb{N})$. De ahí la necesidad de varios documentos y una larga exposición. A pesar de ser reducido y muy simple, ya se podrán empezar a observar cuestiones importantes, que se espera que sirvan como invitación a realizar el esfuerzo de leer la serie entera.\\
	
	\noindent
	2) Explicar, en un capítulo introductorio, el por qué de la necesidad de añadir este documento, y uno más al final, para complementar la TPI\footnote{Transferencia de Pares Ilimitada}. La técnica de la DI\footnote{Diagonalización Inversa: variante pelea de colegio}. Admito haberme quedado estupefacto al observar, como sin saberlo, algunas personas proponen la DI como único argumento sólido para poder negar la TPI. La DI en su día fue rechazada como `crankery', y la TPI fue creada usando ese feedback. Se da el curioso fenómeno que son complementarias. No se puede negar una sin autorizar la otra.\\
	
	\noindent
	3) Un capítulo apéndice, para dejar constancia de referencias, comentar reacciones a la TPI y ya que se saca el tema de las paradojas híbridas, comentar mis hipótesis al respecto, para ofrecer pistas sobre caminos con los que ampliar estos trabajos. Optativo, solo para curiosos.
	
					
\end{abstract}
		
	\addtocontents{toc}{\hspace{-7.5mm} \textbf{Capítulos}}
	\addtocontents{toc}{\hfill \textbf{P\'agina} \par}
	\addtocontents{toc}{\vspace{-2mm} \hspace{-7.5mm} \hrule \par}
	
	\tableofcontents
	
	\part{Part I: English}
	
	\chapter{Introduction}

	\noindent
	Have you read the abstract? Make sure you have before continuing. People keep telling me they don't understand where I'm going with this. This document was created to respond to the reactions to the TPI... and to give you something that starts to provide data on the hypothesis of the existence of hybrid paradoxes. THIS IS NOT THE COUNTEREXAMPLE. It is just a simple example to encourage you to read a much more complex and lengthy one.\\\\
	
	\noindent
	I can't help it, I write a bit disorderly. I don't know how to solve the problem of you thinking that what I show in each document is ALL I have, and if I try to warn you about this, you say I'm disorganized. Not even this series of documents is EVERYTHING I have. The TPI document has 60 pages. I tried to save time with an internet person who couldn't read Spanish, presenting only what is an 'imperfect injection', without examples, and the definition of the TPI. The result was that person did EVERYTHING I beg you not to do in the introduction. I will attach the document that person sent me 'anonymously' in the appendix chapters. You can't see our conversations, where that person clearly talks about 'rotating sets' and indicates that I would be able to see it if I had a formal education that took all these foolishness out of my head. But 'rotating sets' are my own argument in the DI. The one that was denied to me, and so I built the TPI.\\\\
	
	\noindent
	Remember that I am NOT a mathematician. Still, I have a letter from a professor\footnote{Text in the appendix chapter} saying that him have been unable to find a flaw. If the format or rigor were a REAL flaw, beyond the initial tantrum it may provoke, he would have mentioned it. Well, in fact... he did mentioned it... and still don't consider it a serious flaw. PRECISELY, I am tired of asking for help to be able to write all my material in a more conventional way, and to be able to check it more deeply and expand it. In the absence of resources and qualified people, the only thing left to us is the scientific, philosophical, and moral imperative that truth is more valuable than a lie with a comfortable presentation. At least, in mathematics, I hope. I have gone above and beyond to give my best.\\\\
	
	\noindent
	I keep warning that in front of a blackboard it is more enjoyable, because I can skip unnecessary assumptions and adapt, studying and evaluating the direct feedback from each person.\\\\
	
	\noindent
	Every idea, every concept, has been defended in forums, and after several feedbacks and corrections, they have been checked and considered correct. Every result as well. Except for the point that forces me to add the DI, the interpretation of infinite intersections, which I considered unnecessary after the concept of 'rotating sets' was shot down. We will see it in future documents, or you may have already started to forge it in your mind, in order to deny the TPI, if you have read the COMPLETE document. I keep saying it but it seems it is not being heard (read carefully): if you DECIDE that the subsets of the partition of the ORIGIN SET in the TPI never even come close to the cardinality of the Domain set, that indicates that you firmly believe that there exists a set different from the empty set, subtracting from Domain X Domain the largest subset (NWSP) possible, or ALL of them (remember they are nested). After having tried it with ALL the disjoint subsets of the Origin set in each relation. Although you will be unable to mention a single element within it. This will lead you to IGNORE the UNDENIABLE final result of emptiness, and to rely on the fact that for each subset we always leave an infinite set of elements from NWSP uncovered. A triad of inseparable decisions, as I have been able to confirm in the reactions of several people. If that "leftover" set exists in some strange way, we can choose one or more elements from it... and that is the basis of DI (which we will see in future documents). In DI, the set with supposed cardinality $\aleph_{1}$ is the one that has trouble `eliminating' all of the elements. In DI, we decide to ignore the final empty intersection result, which occurs in the same terms but with the roles reversed, and we only focus on the inability of the set with cardinality $\aleph_{1}$ to get rid of a simple set with cardinality $\aleph_{0}$.\\\\
	
	\noindent
	If you are not going to bother contacting me to ask me questions, or if you are going to read it quickly, skipping parts\footnote{READ THE ABSTRACT!!}, etc... please don't even try. YOU ALREADY KNOW that I have flaws as a writer, getting angry about it instead of sending me an email asking me... I don't know how to describe it. Reading it quickly will lead you to come up with real nonsense (too many previous experiences)... One person underlined almost an entire paragraph telling me that a specific property didn't necessarily work on infinite sets. The sentence that was left ununderlined, right at the beginning of the same paragraph, said something like: "...this only applies to sets of finite cardinality". ASK, read with some care and patience, and if you get angry, remember that I have double-checks, experienced people who can't find the mistake and that I have offered to do it in person thousands of times with a blackboard, and they call me arrogant for it. That doubt or perception of error could be resolved in SECONDS, and there are more than two mathematicians who do NOT consider it an error. Especially with the presence of more mathematicians, we could leave arguments from authority aside, since you could see their faces, not mine. I DON'T MIND ANSWERING A MILLION TIMES, I'm the first non-mathematician here. What bothers me is when communication is cut off due to real nonsense. Sometimes with arguments unworthy of a mathematician. Things like saying that a counterexample does not nullify a theorem, since one must explain separately WHY each of the existing proofs fails. Anyway...\\\\
	
	\noindent
	This week I was able to check something that was causing me brutal frustration. I didn't think it was such a complex argument to understand, but 'someone' has already confirmed to me that it is correct. Even that person did not understand why I insisted on the question, and repeated several times that it was correct. As always, working with anonymous people in forums, with whom I cannot continue working when communication is cut off, and months go by until I find another person.\\\\


	\noindent
	The idea is simple, both constructions, if well constructed, would be equivalent to an impossible injective relationship between $P(\mathbb{N})$ and $\mathbb{N}$. Both the TPI (Unlimited Pair Transfer) and the DI (Inverse Diagonalization) need property A, but in complementary ways. A triad of interpretations about the type of infinite intersections presented in the TPI document. Affirming A would consist of:\\\\
	(A.1): We can ignore the final result of empty. I say ignore because it cannot be denied. In both cases, the infinite intersection has an UNDENIABLE result of empty.\\\\
	(A.2): There is a set that can be empty or not: the 'rotating set'. We are unable to mention a single element within the set because we know with absolute certainty that EVERY ONE of its elements is eliminated from it, from a concrete and calculable step of the infinite intersection. The fact that we are unable to mention a single element within it, WE ARE ALSO GOING TO IGNORE IT... arguing observation A.3.\\\\
	(A.3): The 'rotating set', NWSP in the case of the TPI, the Packs\footnote{It seems silly to remember, but given previous experiences: we will see what the Packs are in future documents} of unique naturals in the DI, always has an infinite cardinal in each step, of the infinite terms of the infinite intersection. This is also undeniable. But we decide NOT TO IGNORE THIS, to say that no subset (subsets of the Origin set in the TPI, families of pairs from the Domain in the DI) ever obliges us to empty the objective set (NWSP or The Packs). I have even been told that this is indicative of cardinal superiority. Following that idea, the Packs that are never emptied would indicate that $\mathbb{N}$ has a strictly higher cardinality than $P(\mathbb{N})$. :D.\\\\
	
	\noindent
	Negating A would mean:\\\\
	(NOT\_A.1): We cannot deny the final result of empty. This would mark an infimum value\footnote{A value that may not belong to the infinite series of values, but NOTHING greater than it is unattainable by the decreasing series of values} for any possible state of NWSP or The Packs, that doesn't even have a single element. Let us not forget that the states of NWSP and The Packs are nested sets that strictly contain each other. WE ARE NOT GOING TO IGNORE THIS.\\\\
	(NOT\_A.2): It is stupid to say that a set has a cardinal greater than 0 and be unable to name a single element within it. Not because we don't know it, but because we are absolutely certain that ALL its elements, at some point, cease to belong to the set. WE ARE NOT GOING TO IGNORE THIS.\\\\
	(NOT\_A.3): WE CAN IGNORE that the set we want to empty has a cardinality of $\infty$ in each term of the infinite intersection. As I say in the TPI document, both in the concept of limit and in the example of Achilles and the turtle, the idea of a distance that we can always reduce is used. And it is ignored, without any complex or problem, that every $\epsilon$ or $\delta$, less than any chosen distance, always differs from the true point under study by an infinite number of points. It is not something 'new' in mathematics.\\\\
	
	\noindent
	\textit{Even both infinite intersections, in both constructions, have only $\aleph_{0}$ terms.}\\\\

	\noindent
	I start by saying that it is possible to construct the equivalent of an impossible injective relation according to Cantor's Theorem:\\
	a) For A=true, constructing the DI is possible.\\
	b) For A=false, constructing the TPI is possible.\\
	
	\noindent
	The interpretation of the infinite intersection can be ambiguous. But for each possible interpretation, there exists an impossible injective relation according to Cantor's Theorem. ONE OF THE TWO MUST BE WELL CONSTRUCTED. Since in both, the only doubtful point is A. Therefore: ONE EXISTS. There is a counterexample to the theorem.\\\\
	
	\noindent
	They denied me the DI by saying that the empty set was not an infimum, but it was an EFFECTIVE result. My Packs emptied. It was impossible for me to have even ONE element, per Pack, with which to construct the unique image of each element in the Domain. None survived the infinite process of discarding from the DI (we will see this in another document). It was not possible to construct the injective relation. I didn't have a SINGLE ONE... but of course, if we apply the same idea to NWSP... if NWSP empties out effectively, that means that WSP is (Domain X Domain): an injective relation by Cantorian definition. If NWSP does not empty, I do not have an injective relation according to TPI (ignoring for now the option of the union of all disjoint subsets of the source set). But if NWSP does not empty, then the Packs don't either, and I would have unique options, within each Pack, to construct an injective relation. It's a dead end: a siege. All possibilities are covered. One cannot be denied without authorizing the other.\\\\
	
	\noindent
	Both results are contradictory ONLY if we consider $\aleph_{0}$ and $\aleph_{1}$ to be distinct cardinals. Exchanging the cardinal role between two sets with the same infinite cardinal is common. It is possible to construct relations that make one appear greater than the other, and vice versa.\\\\
	
	\noindent
	Let's consider the natural numbers ($\mathbb{N}$) and the even numbers ($\mathbb{P}$, chosen letter from its name in Spanish):\\\\
	$f_{1}: \mathbb{P} \longrightarrow \mathbb{N}$\\
	$f_{1}(p) = \{p, p+1\}$\\
	*\textit{We have two natural numbers for each even number.}\\\\
	$f_{2}: \mathbb{N} \longrightarrow \mathbb{P}$\\
	$f_{2}(n) = \{n * 10^{4}, (n * 10^{4})+2, (n * 10^{4})+4, ... , (n * 10^{4})+9998 \}$\\
	*\textit{We have 5000 even numbers for each natural number.}\\\\
	
	\noindent
	The DI and TPI are not contradictory, both can be constructed, PRECISELY, because $\aleph_{0}$ was ALWAYS equal to $\aleph_{1}$. That's why the roles of the sets involved in the infinite intersections can be exchanged.\\\\
	
	\noindent
	That's why it's necessary to add the DI, to complete the siege. Although I didn't expect it, the truth is that it was rejected as 'crankery' in its day and now it's being used to deny TPI. The simple example we're going to see below is THAT, a simple example. It shows a siege technique, which is to change the definition of ALL possible sets involved, to see that once changed, the problematic set does not generate any problem, being the same set.\\\\ 
	
	\noindent
	In the TPI, applied to $P(\mathbb{N})$ and $\mathbb{N}$, we will replace the subsets of $\mathbb{N}$ with $SNEFs$ and $SNEIs$. This simple change disables Cantor's proof in various ways, one of which is allowing the construction of a TPI. And as I have seen that things are denied without thinking twice, as long as the Theorem remains alive, even falling into circular arguments, where the theorem is used as a premise of the theorem itself... The DI is also necessary, not just the change in the definition of the sets. The TPI is so well constructed that the only possibility for Cantor's Theorem to remain alive is to resort to an old work OF MINE that was considered 'crankery' in its day. Living to see the arguments that two different mathematicians can come up with. Yes, I am disappointed. I warn them, even if they choose that path, and their solution is to cut off communications without facing their own judgments.\\\\
	
	\noindent
	I know you're going to get angry at the first glance of the simple example. I've gotten some very strange reactions. From accusing me of having chosen it 'too well', to attacking its simplicity. The first one is that I didn't knew mathematical generalizations could have exceptions. The second one is extremely serious: yes, it's too simple, and statistically, a couple of observations that are equally simple to see are going to be overlooked. So please read it twice... if you stop halfway because of that virulent reaction, remember to finish the first chapter, just the first... and you'll understand what I'm talking about. If you think I can't reproduce the phenomenon for more complex sets, you wouldn't actually have found a flaw, you'd just be taking a step back. I would have already demonstrated to you that the phenomenon is possible in one case... and opened up the question of whether there might be more than one. I understand your time is valuable, but I'm not coming empty-handed: I offer you a little crack in the theorem, simple, but a little crack. I'm telling you I can do it for more complex sets. I have references from a professor. I have mathematicians using my own work to deny ANOTHER one of my works (and vice versa). I have every point of the two constructions checked by, at least, two different people. Unofficially, because everyone's time is valuable, and I'm just a supposed 'crankery'. I HAVE DONE MY WORK. Are you going to do yours as guardians of mathematical knowledge? Even before burying yourselves in more text, I try to offer you things to show that the effort is worth it.\\\\
	
	\noindent
	That being said, and with the abstract READ. Let's begin.\\\\
	\chapter{The Example}

	Showing the example, tables and definitions

 





 







	\chapter{NONE obvious properties: as I could experiment}

	Surprisingly this next mathematical facts are not obvious. But they are still FACTS.

	
	\part{Parte II: Español}
	
	\chapter{Introducción}

	\noindent
	¿Has leído el abstract? Asegúrate primero de haberlo hecho. Luego la gente me dice que no entiende a donde quiero llegar. Este documento se ha creado para responder a las reacciones de la TPI... y de paso daros algo que comience a aportar datos sobre la hipótesis de la existencia de las paradojas híbridas. NO ES EL CONTRAEJEMPLO. Solo es un ejemplo sencillo para daros ánimos a leer uno mucho más complejo y largo.\\\\
	
	\noindent
	No puedo evitarlo, escribo un poco desordenado. No sé como resolver el problema de que creáis que lo que enseño en cada documento, es lo ÚNICO que tengo, y si trato de advertirlo me decís que soy desordenado. Ni siquiera esta serie de documentos es TODO lo que tengo. El documento de la TPI tiene 60 páginas. Intenté ahorrármelas con una persona de internet, alguien que no podía leer español, presentando solo lo que es una `inyección imperfecta', sin ejemplos, y la definición de la TPI. El resultado fue que hizo TODO lo que suplico que no se haga en su introducción. Pondré el documento adjunto que me mandó `anónimo' en los capítulos apéndices. En el no se pueden ver nuestras conversaciones, donde habla claramente de `conjuntos rotativos' y me indica que YO sería capaz de verlo si hubiese tenido una educación formal que me quitase todas estas tonterías de la cabeza. Pero es que los `conjuntos rotativos' son mi propio argumento en la DI. El que me negaron y por eso construí la TPI.\\\\
	
	\noindent
	Recordemos que NO soy matemático. Aún así tengo una carta de un catedrático\footnote{Texto en el capítulo apéndice} diciendo que ha sido incapaz de encontrar un fallo. Si el formato o el rigor, fuesen un fallo REAL, más allá de la rabieta inicial que os puede provocar, lo hubiese mencionado. Bueno, de hecho... lo menciona... y aún así no lo considera un fallo grave. PRECISAMENTE estoy harto de pedir ayuda para poder escribir todo mi material, de una forma más convencional, y poder chequearlo de una forma más profunda y expandirlo. Ante esa ausencia de recursos y personas cualificadas, lo único que nos queda es el imperativo científico, filosófico y moral, que una verdad es más valiosa que una mentira con una presentación cómoda. Al menos, en matemáticas, espero. Yo he ido más allá de dar lo mejor de mi mismo.\\\\
	
	\noindent
	No paro de advertir que delante de una pizarra es más ameno, porque me puedo saltar suposiciones innecesarias, y adaptarme, estudiando y evaluando el feedback directo de cada persona. \\\\
	
	\noindent
	Cada idea, cada concepto, ha sido defendido en foros, y tras varios feedbacks y correcciones, han sido chequeados y considerados correctos. Cada resultado también. Excepto el punto que obliga a añadir la DI, la interpretación de las intersecciones infinitas, lo cual consideraba innecesario después de que me tumbasen el concepto de `conjuntos rotativos'. Ya lo veremos en documentos futuros, o puede que tu ya hayas empezado a fraguarlo en tu mente, para poder negar la TPI, si has leído el documento COMPLETO. Me harto de decirlo pero parece ser que no se escucha (lee con cuidado): si DECIDES que los subconjuntos de la partición del CONJUNTO ORIGEN en la TPI, jamás llegan ni a estar cerca del cardinal del conjunto Dominio, eso indica que crees firmemente que existe UN conjunto diferente del vacío, restando a Dominio X Dominio, el mayor subconjunto (NWSP) posible, o TODOS ellos ( recordemos que son anidados ). Después de haberlo intentando con TODOS los subconjuntos disjuntos del conjunto Origen en cada relación. A pesar de que serás incapaz de mencionar un solo elemento dentro de él. Eso te llevará a IGNORAR el INNEGABLE resultado final de vacío, y a apoyarte en que para cada subconjunto siempre nos dejamos sin abarcar un conjunto infinito de elementos de NWSP. Una triada de decisiones inseparables, como he podido comprobar en las reacciones de varias personas. Si ese conjunto `sobrante' existe de alguna forma extraña, podemos escoger uno o más elementos de él... y en eso se basa la DI (que ya veremos en documentos futuros). En la DI al que le cuesta `eliminar' todos los elementos es al conjunto con supuesto cardinal $\aleph_{1}$. En la DI decidimos ignorar el resultado final de vacío de la intersección, que se produce en los mismos términos pero con los papeles invertidos, y solo nos fijamos en la impotencia del conjunto con cardinal $\aleph_{1}$ para deshacerse de un simple conjunto con cardinal $\aleph_{0}$.\\\\
	
	\noindent
	Si no te vas a molestar en contactarme para preguntarme dudas, o lo vas a leer rápido, saltándote partes\footnote{LÉETE EL ABSTRACT!!}, etc... Por favor, ni lo intentes. YA SABES que tengo defectos como escritor, cabrearse por ello, en vez de enviarme un e-mail preguntándome... no sé como calificarlo. Leerlo rápido te va a llevar a plantearte auténticas tonterías (demasiadas experiencias previas)... Una persona me subrayó casi un párrafo entero diciéndome que una propiedad concreta no tenía por qué funcionar en conjuntos infinitos. La frase que se dejó sin subrayar, justo al inicio del mismo párrafo, decía algo así como: "...esto aplica solo a conjuntos de cardinalidad finita". PREGUNTA, lee con cierto cariño y paciencia, y si te cabreas, recuerda que tengo dobles chequeos, gente potente que no encuentra el fallo y que me he ofrecido a hacerlo en persona miles de veces con una pizarra, y me llaman prepotente por ello. Esa duda, o apreciación de error, la podría resolver en SEGUNDOS, y hay más de dos matemáticos que NO lo consideran un error. Sobre todo con la presencia de más matemátic@s, podríamos dejar los argumentos de autoridad a un lado, ya que podrías ver sus caras, no la mía. NO ME IMPORTA RESPONDER UN MILLÓN DE VECES, yo soy el primero que no es matemático aquí. Lo que me molesta es que se corte la comunicación por auténticas tonterías. A veces con argumentos indignos de un matemático. Cosas como decir que un contraejemplo no anula un teorema, ya que se debe explicar, aparte, el POR QUÉ falla cada una de las demostraciones que existen. En fin...\\\\
	
	\noindent
	Esta semana he podido chequear algo que me producía una frustración brutal. No creía que fuese un argumento tan complejo de entender, pero 'alguien´ ya me ha confirmado que es correcto. Incluso esa persona no entendía por qué insistía en la pregunta, y me repitió varias veces que era correcto. Como siempre, trabajando con gente anónima en foros, con la que no puedo seguir trabajando cuando cortan la comunicación, y pasan meses hasta que vuelvo a encontrar a otra persona.\\\\
	
	\noindent
	La idea es simple, las dos construcciones, si estuviesen bien construidas, serían equivalentes a una relación inyectiva imposible entre $P(\mathbb{N})$ y $\mathbb{N}$. Tanto la TPI (Transferencia de Pares Ilimitada) como la DI (Diagonalización Inversa), necesitan la propiedad A, pero de formas complementarias. Una triada de interpretaciones sobre el tipo de intersecciones infinitas que se presentan en el documento de la TPI. Afirmar A consistiría en:\\\\
	(A.1): Podemos ignorar el resultado final de vacío. Digo ignorar porque no se puede negar. En ambos casos la intersección infinita tiene un resultado INNEGABLE de vacío.\\\\
	(A.2): Hay un conjunto que puede ser vacío o no: el 'conjunto rotativo´. Somos incapaces de mencionar un solo elemento dentro del conjunto, porque sabemos con absoluta seguridad, que CADA UNO de sus elementos es eliminado de él, a partir de un paso concreto y calculable de la intersección infinita. El hecho de no ser capaces de mencionar un solo elemento dentro de él, LO VAMOS A IGNORAR TAMBIÉN... argumentando la observación A.3.\\\\
	(A.3): El 'conjunto rotativo´, NWSP en el caso de la TPI, los Packs\footnote{Parece una tontería recordarlo, pero dadas experiencias previas: ya veremos que son los Packs en futuros documentos} de naturales únicos en la DI, siempre tiene un cardinal infinito en cada paso, de los infinitos términos de la intersección infinita. Esto es innegable también. Pero decidimos NO IGNORAR esto, para decir que ningún subconjunto (subconjuntos del conjunto Origen en la TPI, familias de pares del Dominio en la DI) nos obliga jamás a vaciar el conjunto objetivo (NWSP o Los Packs). Incluso me han llegado a afirmar que eso es indicativo de superioridad cardinal. Siguiendo esa idea, los Packs que nunca se vacían indicarían que $\mathbb{N}$ tiene un cardinal estrictamente superior al de $P(\mathbb{N})$.:D.\\\\
	
	\noindent
	Negar A, significaría:\\\\
	(NOT\_A.1): No podemos negar el resultado final de vacío. Esto marcaría un valor infimum\footnote{Un valor que igual no pertenece a la serie infinita de valores, pero NADA mayor que él, es inalcanzable por la serie de valores decrecientes} para todo posible estado de NWSP, o los Packs, que ni siquiera tiene un solo elemento. No olvidemos que los estados de NWSP, y los Packs, son conjuntos anidados que se contienen estrictamente los unos a los otros. NO LO VAMOS A IGNORAR.\\\\
	(NOT\_A.2): Es estúpido decir que un conjunto tiene un cardinal mayor que $0$, y ser incapaz de nombrar un sólo elemento dentro de él. No porque lo desconozcamos, sino porque estamos absolutamente seguros que TODOS sus elementos, en algún momento, dejan de pertenecer al conjunto. NO LO VAMOS A IGNORAR.\\\\
	(NOT\_A.3): PODEMOS IGNORAR que el conjunto que queremos vaciar, tenga cardinal $\infty$ en cada término de la intersección infinita. Como digo en el documento de la TPI, tanto en el concepto de límite, como en el ejemplo de Aquiles y la tortuga, se usa la idea de una distancia que siempre podemos reducir. Y se ignora, sin ningún complejo o problema, que cada $\epsilon$ o $\delta$, menor a cualquier distancia escogida, siempre dista del verdadero punto a estudiar una cantidad infinita de puntos. No es algo 'nuevo´ en matemáticas.\\\\
	
	\noindent
	\textit{Incluso ambas intersecciones infinitas, en ambas construcciones, tienen solo $\aleph_{0}$ términos.}\\\\
	
	\noindent
	Parto de decir que es posible construir el equivalente a una relación inyectiva imposible, según el Teorema de Cantor:\\
	a) Para A=verdadero, construir la DI es posible\\
	b) Para A=falso, construir la TPI es posible\\\\
	
	\noindent
	La interpretación de la intersección infinita puede ser ambigua. Pero para cada posible interpretación, existe una relación inyectiva imposible según el Teorema de Cantor. UNA DE LAS DOS DEBE ESTAR BIEN CONSTRUIDA. Ya que en ambas, el único punto dudoso es A. Por lo tanto: UNA EXISTE. Existe UN contraejemplo del Teorema.\\\\
	
	\noindent
	La DI me la negaron diciendo que el vacío no era un infimum, sino que era un resultado EFECTIVO. Mis Packs se vaciaban. Me resultaba imposible tener UN SOLO elemento, por Pack, con el que construir la imagen única de cada elemento del Dominio. No sobrevivía NINGUNO al infinito proceso de descarte de la DI (ya lo veremos en otro documento). No era posible construir la relación inyectiva. No tenía NI UNO SOLO... pero claro, si aplicamos la misma idea a NWSP... si NWSP se vacía de forma efectiva, eso significa que WSP es (Dominio X Dominio): una relación inyectiva por definición cantoriana. Si NWSP no se vacía, no tengo una relación inyectiva según la TPI (ignorando por ahora la opción de la unión de todos los subconjuntos disjuntos del conjunto origen). Pero si NWSP no se vacía, los Packs tampoco, y tendría opciones únicas, dentro de cada Pack, para construir una relación inyectiva. Es un callejón sin salida: un asedio. Todas las posibilidades están cubiertas. No se puede negar una sin autorizar la otra.\\\\
	
	\noindent
	Ambos resultados son contradictorios SOLO si consideramos que $\aleph_{0}$ y $\aleph_{1}$ son cardinales distintos. Intercambiar el papel cardinal entre dos conjuntos con el mismo cardinal infinito, es algo común. Es posible construir relaciones que hagan a uno aparentar ser mayor que el otro, y viceversa.\\\\
	
	\noindent
	Consideremos los números naturales ($\mathbb{N}$) y los números pares ($\mathbb{P}$, letra escogida de su nombre en español):\\\\
	$f_{1}: \mathbb{P} \longrightarrow \mathbb{N}$\\
	$f_{1}(p) = \{p, p+1\}$\\
	*\textit{Tenemos dos números naturales por cada número par.}\\\\
	$f_{2}: \mathbb{N} \longrightarrow \mathbb{P}$\\
	$f_{2}(n) = \{n*10^{4}, (n*10^{4})+2, (n*10^{4})+4, ... , (n*10^{4})+9998 \}$\\
	* \textit{Tenemos 5000 números pares por cada número natural.}\\\\
	
	\noindent
	La DI y la TPI no son contradictorias, las dos se pueden construir, PRECISAMENTE, porque $\aleph_{0}$ SIEMPRE fue igual a $\aleph_{1}$. Por eso se pueden intercambiar los papeles de los conjuntos implicados en las intersecciones infinitas.\\\\
	
	\noindent
	Por eso es necesario añadir la DI, para completar el asedio. Aunque no me lo esperaba, la verdad. En su dia fue rechazada por 'crankery´ y ahora la usan para poder negar la TPI. El ejemplo sencillo que vamos a ver a continuación es ESO, un ejemplo sencillo. Muestra una técnica de asedio, que es cambiar la definición de TODOS los posibles conjuntos implicados, para ver que una vez cambiadas, el conjunto problemático no genera ningún problema, siendo el mismo conjunto. En la TPI, aplicada a $P(\mathbb{N})$ y $\mathbb{N}$, cambiaremos los subconjuntos de $\mathbb{N}$ por $SNEFs$ y $SNEIs$. Ese simple cambio deshabilita la demostración de Cantor, de varias formas, una de ellas es permitir construir una TPI. Y como he podido comprobar que las cosas se niegan sin pensar dos veces, con tal de que el Teorema siga vivo, incluso cayendo en argumentos circulares, donde se usa el teorema como premisa del propio teorema... La DI es necesaria también, no solo el cambio de definición de los conjuntos. La TPI está tan bien construida, que la única posibilidad de que el Teorema de Cantor siga vivo es tirar de un antiguo trabajo MIO que fue considerado 'crankery´ en su dia. Vivir para ver los argumentos que pueden llegar a usar DOS matemáticos diferentes. Si, estoy defraudado. Les aviso, aún así escogen ese camino, y su solución es cortar comunicaciones sin enfrentarse a sus propios juicios.\\\\
	
	\noindent
	Sé que te vas a cabrear en el primer vistazo al ejemplo simple. He obtenido reacciones muy extrañas. Desde acusarme de haberlo escogido 'demasiado bien´, hasta atacar su simpleza. Lo primero es que no sabía que las generalizaciones matemáticas podían tener excepciones. Lo segundo es sangrante: sí, es demasiado sencillo, y estadísticamente, se te van a pasar por alto un par de observaciones igualmente simples de ver. Así que lee dos veces por favor... si te paras a medias, por esa reacción virulenta, recuerda acabar el primer capítulo, solo el primero... y entenderás de lo que hablo. Si te piensas que no puedo reproducir el fenómeno para conjuntos más complejos, en realidad no habrías encontrado un fallo, sólo estarías dando un paso atrás. Ya te habría demostrado que el fenómeno es posible en un caso... y abierto la duda sobre si es posible que exista más de uno. Entiendo que tu tiempo es oro, pero no vengo con las manos vacías: te ofrezco una grieta en el teorema, simple, pero una grieta. Te digo que lo puedo hacer para conjuntos más complejos. Tengo referencias de un catedrático. Tengo matemáticos usando mi propio trabajo para negar OTRO de mis trabajos (y viceversa). Tengo cada punto de las dos construcciones, chequeado por, al menos, dos personas diferentes. De forma no oficial, porque el tiempo de todo el mundo es oro, y yo solo soy un supuesto 'crankery´. YO HE HECHO MI TRABAJO. ¿Vais a hacer el vuestro como guardianes del conocimiento matemático? Incluso antes de enterraros en más texto, procuro ofreceros cosas para demostrar que el esfuerzo merece la pena.
	\\\\ 
	
	\noindent
	Dicho esto, y con el abstract LEÍDO. Comencemos.
	
	





	\input{CAPITULOS/Part_II_Español/CAP_II_El_Ejemplo/Cap_II_ESP_EL_Ejemplo.tex}
	\chapter{Propiedades NO obvias: como pude experimentar}

	Sorpresivamente, las siguientes propiedades matemáticas NO son obvias. Pero siguen siendo HECHOS.
	
\end{document}